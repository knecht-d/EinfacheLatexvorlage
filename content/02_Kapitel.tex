\chapter{Kapitel}\label{sec:kaptiel}

\blindtext

\begin{figure}[h]
    \centering
    \includegraphics[width=0.5\textwidth]{placeholder}
    \caption{a nice placeholder}
    \label{fig:place}
\end{figure}

\blindtext
Außerdem lässt sich so eine \autoref{fig:place} referenzieren\cite{dreyfus:1980}.
Weitere Informationen siehe \autoref{exk:exkurs}

\begin{myExkurs}[H]{Exkurse}{exk:exkurs}
    Exkurse dienen danzu weiterführende Informationen zur Verfügung zu stellen.
\end{myExkurs}

\begin{minipage}[c]{\textwidth}
    \lstinputlisting[
        caption=Code-Beispiel aus \cite{mustermann:2012},
        label=lst:hello
    ]{hello-world.java}
\end{minipage}

Auch lässt sich das \autoref{lst:hello} referenzieren.
Wobei folgendes zu beachten ist:
\blindtext

\lstinputlisting[
    caption=Python-Code,
    label=Python-Code
    language=Python
]{quicksort.py}